% Options for packages loaded elsewhere
\PassOptionsToPackage{unicode}{hyperref}
\PassOptionsToPackage{hyphens}{url}
%
\documentclass[
]{article}
\usepackage{amsmath,amssymb}
\usepackage{lmodern}
\usepackage{iftex}
\ifPDFTeX
  \usepackage[T1]{fontenc}
  \usepackage[utf8]{inputenc}
  \usepackage{textcomp} % provide euro and other symbols
\else % if luatex or xetex
  \usepackage{unicode-math}
  \defaultfontfeatures{Scale=MatchLowercase}
  \defaultfontfeatures[\rmfamily]{Ligatures=TeX,Scale=1}
\fi
% Use upquote if available, for straight quotes in verbatim environments
\IfFileExists{upquote.sty}{\usepackage{upquote}}{}
\IfFileExists{microtype.sty}{% use microtype if available
  \usepackage[]{microtype}
  \UseMicrotypeSet[protrusion]{basicmath} % disable protrusion for tt fonts
}{}
\makeatletter
\@ifundefined{KOMAClassName}{% if non-KOMA class
  \IfFileExists{parskip.sty}{%
    \usepackage{parskip}
  }{% else
    \setlength{\parindent}{0pt}
    \setlength{\parskip}{6pt plus 2pt minus 1pt}}
}{% if KOMA class
  \KOMAoptions{parskip=half}}
\makeatother
\usepackage{xcolor}
\IfFileExists{xurl.sty}{\usepackage{xurl}}{} % add URL line breaks if available
\IfFileExists{bookmark.sty}{\usepackage{bookmark}}{\usepackage{hyperref}}
\hypersetup{
  pdftitle={Laboratorio-1.R},
  pdfauthor={zupap},
  hidelinks,
  pdfcreator={LaTeX via pandoc}}
\urlstyle{same} % disable monospaced font for URLs
\usepackage[margin=1in]{geometry}
\usepackage{color}
\usepackage{fancyvrb}
\newcommand{\VerbBar}{|}
\newcommand{\VERB}{\Verb[commandchars=\\\{\}]}
\DefineVerbatimEnvironment{Highlighting}{Verbatim}{commandchars=\\\{\}}
% Add ',fontsize=\small' for more characters per line
\usepackage{framed}
\definecolor{shadecolor}{RGB}{248,248,248}
\newenvironment{Shaded}{\begin{snugshade}}{\end{snugshade}}
\newcommand{\AlertTok}[1]{\textcolor[rgb]{0.94,0.16,0.16}{#1}}
\newcommand{\AnnotationTok}[1]{\textcolor[rgb]{0.56,0.35,0.01}{\textbf{\textit{#1}}}}
\newcommand{\AttributeTok}[1]{\textcolor[rgb]{0.77,0.63,0.00}{#1}}
\newcommand{\BaseNTok}[1]{\textcolor[rgb]{0.00,0.00,0.81}{#1}}
\newcommand{\BuiltInTok}[1]{#1}
\newcommand{\CharTok}[1]{\textcolor[rgb]{0.31,0.60,0.02}{#1}}
\newcommand{\CommentTok}[1]{\textcolor[rgb]{0.56,0.35,0.01}{\textit{#1}}}
\newcommand{\CommentVarTok}[1]{\textcolor[rgb]{0.56,0.35,0.01}{\textbf{\textit{#1}}}}
\newcommand{\ConstantTok}[1]{\textcolor[rgb]{0.00,0.00,0.00}{#1}}
\newcommand{\ControlFlowTok}[1]{\textcolor[rgb]{0.13,0.29,0.53}{\textbf{#1}}}
\newcommand{\DataTypeTok}[1]{\textcolor[rgb]{0.13,0.29,0.53}{#1}}
\newcommand{\DecValTok}[1]{\textcolor[rgb]{0.00,0.00,0.81}{#1}}
\newcommand{\DocumentationTok}[1]{\textcolor[rgb]{0.56,0.35,0.01}{\textbf{\textit{#1}}}}
\newcommand{\ErrorTok}[1]{\textcolor[rgb]{0.64,0.00,0.00}{\textbf{#1}}}
\newcommand{\ExtensionTok}[1]{#1}
\newcommand{\FloatTok}[1]{\textcolor[rgb]{0.00,0.00,0.81}{#1}}
\newcommand{\FunctionTok}[1]{\textcolor[rgb]{0.00,0.00,0.00}{#1}}
\newcommand{\ImportTok}[1]{#1}
\newcommand{\InformationTok}[1]{\textcolor[rgb]{0.56,0.35,0.01}{\textbf{\textit{#1}}}}
\newcommand{\KeywordTok}[1]{\textcolor[rgb]{0.13,0.29,0.53}{\textbf{#1}}}
\newcommand{\NormalTok}[1]{#1}
\newcommand{\OperatorTok}[1]{\textcolor[rgb]{0.81,0.36,0.00}{\textbf{#1}}}
\newcommand{\OtherTok}[1]{\textcolor[rgb]{0.56,0.35,0.01}{#1}}
\newcommand{\PreprocessorTok}[1]{\textcolor[rgb]{0.56,0.35,0.01}{\textit{#1}}}
\newcommand{\RegionMarkerTok}[1]{#1}
\newcommand{\SpecialCharTok}[1]{\textcolor[rgb]{0.00,0.00,0.00}{#1}}
\newcommand{\SpecialStringTok}[1]{\textcolor[rgb]{0.31,0.60,0.02}{#1}}
\newcommand{\StringTok}[1]{\textcolor[rgb]{0.31,0.60,0.02}{#1}}
\newcommand{\VariableTok}[1]{\textcolor[rgb]{0.00,0.00,0.00}{#1}}
\newcommand{\VerbatimStringTok}[1]{\textcolor[rgb]{0.31,0.60,0.02}{#1}}
\newcommand{\WarningTok}[1]{\textcolor[rgb]{0.56,0.35,0.01}{\textbf{\textit{#1}}}}
\usepackage{graphicx}
\makeatletter
\def\maxwidth{\ifdim\Gin@nat@width>\linewidth\linewidth\else\Gin@nat@width\fi}
\def\maxheight{\ifdim\Gin@nat@height>\textheight\textheight\else\Gin@nat@height\fi}
\makeatother
% Scale images if necessary, so that they will not overflow the page
% margins by default, and it is still possible to overwrite the defaults
% using explicit options in \includegraphics[width, height, ...]{}
\setkeys{Gin}{width=\maxwidth,height=\maxheight,keepaspectratio}
% Set default figure placement to htbp
\makeatletter
\def\fps@figure{htbp}
\makeatother
\setlength{\emergencystretch}{3em} % prevent overfull lines
\providecommand{\tightlist}{%
  \setlength{\itemsep}{0pt}\setlength{\parskip}{0pt}}
\setcounter{secnumdepth}{-\maxdimen} % remove section numbering
\ifLuaTeX
  \usepackage{selnolig}  % disable illegal ligatures
\fi

\title{Laboratorio-1.R}
\author{zupap}
\date{2024-04-24}

\begin{document}
\maketitle

\begin{Shaded}
\begin{Highlighting}[]
\CommentTok{\# Laboratorio 1}
\CommentTok{\# Alondra Perales}
\CommentTok{\# 14/04/2024}
\CommentTok{\# 2070702}

\CommentTok{\# Gastos totales}
\DecValTok{300} \SpecialCharTok{+} \DecValTok{240} \SpecialCharTok{+} \DecValTok{1527} \SpecialCharTok{+} \DecValTok{400} \SpecialCharTok{+} \DecValTok{1500} \SpecialCharTok{+} \DecValTok{1833}
\end{Highlighting}
\end{Shaded}

\begin{verbatim}
## [1] 5800
\end{verbatim}

\begin{Shaded}
\begin{Highlighting}[]
\NormalTok{celular }\OtherTok{\textless{}{-}} \DecValTok{300}
\NormalTok{celular}
\end{Highlighting}
\end{Shaded}

\begin{verbatim}
## [1] 300
\end{verbatim}

\begin{Shaded}
\begin{Highlighting}[]
\NormalTok{transporte }\OtherTok{\textless{}{-}} \DecValTok{240}
\NormalTok{transporte}
\end{Highlighting}
\end{Shaded}

\begin{verbatim}
## [1] 240
\end{verbatim}

\begin{Shaded}
\begin{Highlighting}[]
\NormalTok{comestibles }\OtherTok{\textless{}{-}} \DecValTok{1527}
\NormalTok{comestibles}
\end{Highlighting}
\end{Shaded}

\begin{verbatim}
## [1] 1527
\end{verbatim}

\begin{Shaded}
\begin{Highlighting}[]
\NormalTok{gimnasio }\OtherTok{\textless{}{-}} \DecValTok{400}
\NormalTok{gimnasio}
\end{Highlighting}
\end{Shaded}

\begin{verbatim}
## [1] 400
\end{verbatim}

\begin{Shaded}
\begin{Highlighting}[]
\NormalTok{alquiler }\OtherTok{\textless{}{-}} \DecValTok{1500}
\NormalTok{alquiler}
\end{Highlighting}
\end{Shaded}

\begin{verbatim}
## [1] 1500
\end{verbatim}

\begin{Shaded}
\begin{Highlighting}[]
\NormalTok{otros }\OtherTok{\textless{}{-}} \DecValTok{1833}
\NormalTok{otros}
\end{Highlighting}
\end{Shaded}

\begin{verbatim}
## [1] 1833
\end{verbatim}

\begin{Shaded}
\begin{Highlighting}[]
\NormalTok{gastos }\OtherTok{\textless{}{-}} \DecValTok{5800} \CommentTok{\# objeto total con la suma de datos}
\NormalTok{gastos}
\end{Highlighting}
\end{Shaded}

\begin{verbatim}
## [1] 5800
\end{verbatim}

\begin{Shaded}
\begin{Highlighting}[]
\CommentTok{\# Gastos durante semestre escolar}
\DecValTok{5800} \SpecialCharTok{+} \DecValTok{5800} \SpecialCharTok{+} \DecValTok{5800} \SpecialCharTok{+} \DecValTok{5800} \SpecialCharTok{+} \DecValTok{5800}
\end{Highlighting}
\end{Shaded}

\begin{verbatim}
## [1] 29000
\end{verbatim}

\begin{Shaded}
\begin{Highlighting}[]
\CommentTok{\# Gastos durante un año}
\DecValTok{29000} \SpecialCharTok{+} \DecValTok{29000}
\end{Highlighting}
\end{Shaded}

\begin{verbatim}
## [1] 58000
\end{verbatim}

\begin{Shaded}
\begin{Highlighting}[]
\NormalTok{gastos }\OtherTok{\textless{}{-}} \FunctionTok{c}\NormalTok{(celular, transporte, comestibles, gimnasio, alquiler, otros)}

\FunctionTok{barplot}\NormalTok{(gastos)}
\end{Highlighting}
\end{Shaded}

\includegraphics{Laboratorio-1_files/figure-latex/unnamed-chunk-1-1.pdf}

\begin{Shaded}
\begin{Highlighting}[]
\NormalTok{gastos\_ordenados }\OtherTok{\textless{}{-}} \FunctionTok{sort}\NormalTok{(gastos, }\AttributeTok{decreasing =} \ConstantTok{TRUE}\NormalTok{)}

\FunctionTok{barplot}\NormalTok{(gastos\_ordenados)}
\end{Highlighting}
\end{Shaded}

\includegraphics{Laboratorio-1_files/figure-latex/unnamed-chunk-1-2.pdf}

\begin{Shaded}
\begin{Highlighting}[]
\CommentTok{\# Parte II Variables}
\CommentTok{\# Problema 1}
\CommentTok{\# Identifique el tipo de variable (cualitativa o cuantitativa) para la lista de preguntas de una encuesta }
\CommentTok{\# aplicada a estudiantes universitarios en una clase de estadistica:}
\CommentTok{\# Fecha de nacimiento (p. Ej., 21/10/1995) : cuantitativa}
\CommentTok{\# Nombre del estudiante: cualitativa}
\CommentTok{\# Edad: cuantitativa}
\CommentTok{\# Direccion de casa: cualitativa}
\CommentTok{\# Numero de telefono: cuantitativa}
\CommentTok{\# Area principal de estudio: cualitativa}
\CommentTok{\# Grado de año universitario: cualitativa}
\CommentTok{\# Puntaje de prueba a mitad de periodo: cuantitativa }
\CommentTok{\# Calificacion general: A, B, C, D. cualitativa}
\CommentTok{\# Tiempo para completar la prueba final de MCF 202: cuantitativa }
\CommentTok{\# Numero de hermanos: cuantitativa}

\CommentTok{\# Problema 2: elija un objeto y obtega una lista de 14 variables, 7 cuantitativas y 7 cualitativas}
\CommentTok{\# Cuantitativas}
\CommentTok{\# Peso}
\CommentTok{\# Altura}
\CommentTok{\# Edad}
\CommentTok{\# Periodo de ovulacion}
\CommentTok{\# Periodo de reproduccion}
\CommentTok{\# Numero de dientes}
\CommentTok{\# Peridoo de lactacion}

\CommentTok{\# Cualitativas}
\CommentTok{\# Clase}
\CommentTok{\# Tipo de sangre}
\CommentTok{\# Familia}
\CommentTok{\# Tipo de reproduccion}
\CommentTok{\# Color}
\CommentTok{\# Olor}
\CommentTok{\# Tipo de pelaje}

\CommentTok{\# Problema 3: considere una variable de investigacion con valores numericos que describen }
\CommentTok{\# fromas electronicas de expresar opiniones de personas: 1= Twitter, 2= Correo electronico ,3= }
\CommentTok{\# mensaje de texto, 4 = facebook, 5= blog ¿ Esta es una variable cualitativa o cuantitativa? }
\CommentTok{\# estas son variables cuantitativas ya que en todas se pueden obtener datos de opiniones, los que }
\CommentTok{\# estan a favor y los que estan en contra el numero total de usuarios entre otras cosas.}

\CommentTok{\# Problema 4: }
\CommentTok{\# Para cada pregunta de ingestigacion, (1) identifique a los individuos de interes}
\CommentTok{\# el grupo o los grupos que estan estudiando, (2) identifique las variable (s) ( la}
\CommentTok{\# caracteristicas sobre la que recopilariamos datos), (3) deterimne si cada variable}
\CommentTok{\# es categorigo cuantitativa  o cualitativa. Explique}

\CommentTok{\# ¿ Cual es la cantidad promedio de horas que los estudiantes de universidades publicas trabajan cada semana?}
\CommentTok{\#ht \textless{}{-} c(10, 14, 12, 18, 23, 15, 6, 9, 14, 24)}
\CommentTok{\# mean(ht)}

\CommentTok{\# ¿Que proporcion de todos los estudiantes universitarios de Mexico estan inscritos en una}
\CommentTok{\# universidad publica?}
\CommentTok{\#EdU \textless{}{-} (230000)}
\CommentTok{\#EdUPrivadas \textless{}{-}(170000)}
\CommentTok{\#EdU {-} EdUPrivadas}

\CommentTok{\# En las universidades publicas, ¿las estudiantes femeninas tienen un promedio de CENEVAL}
\CommentTok{\# mas alto que los estudinates varones? si}

\CommentTok{\# ¿Es mas probable que los atletas universitarios reciban asesoramientos academico}
\CommentTok{\# que los atletas no universitarios? no}

\CommentTok{\# Si reunieramos datos para responder a las preguntas de investigacion anterior, ¿que datos}
\CommentTok{\# podrian analizarse medinate un histograma? ¿Como lo sabes?}
\CommentTok{\# si, porque mediante el histograma se puede observar la cantidad de variables que estas tomando}
\CommentTok{\# para responder dichas preguntas.}
\end{Highlighting}
\end{Shaded}


\end{document}
